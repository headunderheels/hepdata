% Template for PHENIX Physics Paper submission to PRL or PRC or PRD
%
%      [Modified by Brant to include PRD on May 21, 2004]
%      [Modified by Brant to include simple PRL length check on July 22, 2004]
%
%    template4.tex for RevTeX4 (highly preferred over RevTeX3)
%
% Please ask Brant <brant@bnl.gov>, if you have any questions
% about the preparation of your draft.  
%
%   Copyright (c) 2001 The American Physical Society.
%
% This is a template for producing manuscripts for use with REVTEX 4.0
% Copy this file to another name and then work on that file.
% That way, you always have this original template file to use or
% retrieve a fresh copy from:
% https://www.phenix.bnl.gov/phenix/WWW//p/info/dp/000/template
%
% To produce single column, double-spaced printout for review
% and for length check, use the "preprint" style:

% For Phys. Rev. Lett. choose (uncomment) one of:
\documentclass[aps,prl,superscriptaddress,showpacs,floatfix,twocolumn]{revtex4}
%\documentclass[aps,prl,superscriptaddress,showpacs,endfloats,preprint]{revtex4}

% For Phys. Rev. C choose (uncomment) one of:
%\documentclass[aps,prc,superscriptaddress,showpacs,nofootinbib,floatfix,twocolumn]{revtex4}
%\documentclass[aps,prc,superscriptaddress,showpacs,nofootinbib,floatfix,preprint]{revtex4}

% For Phys. Rev. D choose (uncomment) one of:
%\documentclass[aps,prd,superscriptaddress,showpacs,nofootinbib,floatfix,twocolumn]{revtex4}
%\documentclass[aps,prd,superscriptaddress,showpacs,nofootinbib,floatfix,preprint]{revtex4}

% To use the \includegraphics package we usually use:
\usepackage{graphicx}	% Include figure files

% Other packages are also available; for example:
% \usepackage{dcolumn}	% Align table columns on decimal point
% \usepackage{bm}	% bold math

% The text is assumed to average 88 characters per line.
% The contribution of text to the total length is calculated as
% Text-lines = Number-of-88-character lines * (88/55).
% [Note:  If one word spills over onto the next line in preprint
%         format, that contributes 1.6 lines to length estimate!]

% It is recommended to use BibTeX and apsrev.bst for references
% Choosing a journal automatically selects the correct APS
% BibTeX style file (bst file), so only uncomment the line
% below if necessary.
%\bibliographystyle{apsrev}

\begin{document}

%Title of paper
\title{Single Electrons from Heavy Flavor Decays in p+p Collisions 
at $\sqrt{s} = 200$~GeV}

\newcommand{\abilene}{Abilene Christian University, Abilene, TX 79699, USA}
\newcommand{\acadsin}{Institute of Physics, Academia Sinica, Taipei 11529, Taiwan}
\newcommand{\banaras}{Department of Physics, Banaras Hindu University, Varanasi 221005, India}
\newcommand{\barc}{Bhabha Atomic Research Centre, Bombay 400 085, India}
\newcommand{\bnl}{Brookhaven National Laboratory, Upton, NY 11973-5000, USA}
\newcommand{\caucr}{University of California - Riverside, Riverside, CA 92521, USA}
\newcommand{\ciae}{China Institute of Atomic Energy (CIAE), Beijing, People's Republic of China}
\newcommand{\cns}{Center for Nuclear Study, Graduate School of Science, University of Tokyo, 7-3-1 Hongo, Bunkyo, Tokyo 113-0033, Japan}
\newcommand{\columbia}{Columbia University, New York, NY 10027 and Nevis Laboratories, Irvington, NY 10533, USA}
\newcommand{\dapnia}{Dapnia, CEA Saclay, F-91191, Gif-sur-Yvette, France}
\newcommand{\debrecen}{Debrecen University, H-4010 Debrecen, Egyetem t{\'e}r 1, Hungary}
\newcommand{\fsu}{Florida State University, Tallahassee, FL 32306, USA}
\newcommand{\gsu}{Georgia State University, Atlanta, GA 30303, USA}
\newcommand{\hiroshima}{Hiroshima University, Kagamiyama, Higashi-Hiroshima 739-8526, Japan}
\newcommand{\ihepprot}{IHEP Protvino, State Research Center of Russian Federation, Institute for High Energy Physics, Protvino, 142281, Russia}
\newcommand{\isu}{Iowa State University, Ames, IA 50011, USA}
\newcommand{\jinrdubna}{Joint Institute for Nuclear Research, 141980 Dubna, Moscow Region, Russia}
\newcommand{\kaeri}{KAERI, Cyclotron Application Laboratory, Seoul, South Korea}
\newcommand{\kangnung}{Kangnung National University, Kangnung 210-702, South Korea}
\newcommand{\kek}{KEK, High Energy Accelerator Research Organization, Tsukuba, Ibaraki 305-0801, Japan}
\newcommand{\kfki}{KFKI Research Institute for Particle and Nuclear Physics of the Hungarian Academy of Sciences (MTA KFKI RMKI), H-1525 Budapest 114, POBox 49, Budapest, Hungary}
\newcommand{\korea}{Korea University, Seoul, 136-701, Korea}
\newcommand{\kurchatov}{Russian Research Center ``Kurchatov Institute", Moscow, Russia}
\newcommand{\kyoto}{Kyoto University, Kyoto 606-8502, Japan}
\newcommand{\labllr}{Laboratoire Leprince-Ringuet, Ecole Polytechnique, CNRS-IN2P3, Route de Saclay, F-91128, Palaiseau, France}
\newcommand{\lawllnl}{Lawrence Livermore National Laboratory, Livermore, CA 94550, USA}
\newcommand{\losalamos}{Los Alamos National Laboratory, Los Alamos, NM 87545, USA}
\newcommand{\lpc}{LPC, Universit{\'e} Blaise Pascal, CNRS-IN2P3, Clermont-Fd, 63177 Aubiere Cedex, France}
\newcommand{\lund}{Department of Physics, Lund University, Box 118, SE-221 00 Lund, Sweden}
\newcommand{\muenster}{Institut f\"ur Kernphysik, University of Muenster, D-48149 Muenster, Germany}
\newcommand{\myongji}{Myongji University, Yongin, Kyonggido 449-728, Korea}
\newcommand{\nagasaki}{Nagasaki Institute of Applied Science, Nagasaki-shi, Nagasaki 851-0193, Japan}
\newcommand{\newmex}{University of New Mexico, Albuquerque, NM 87131, USA}
\newcommand{\nmsu}{New Mexico State University, Las Cruces, NM 88003, USA}
\newcommand{\ornl}{Oak Ridge National Laboratory, Oak Ridge, TN 37831, USA}
\newcommand{\orsay}{IPN-Orsay, Universite Paris Sud, CNRS-IN2P3, BP1, F-91406, Orsay, France}
\newcommand{\pnpi}{PNPI, Petersburg Nuclear Physics Institute, Gatchina,  Leningrad region, 188300, Russia}
\newcommand{\riken}{RIKEN, The Institute of Physical and Chemical Research, Wako, Saitama 351-0198, Japan}
\newcommand{\rikjrbrc}{RIKEN BNL Research Center, Brookhaven National Laboratory, Upton, NY 11973-5000, USA}
\newcommand{\saispbstu}{Saint Petersburg State Polytechnic University, St. Petersburg, Russia}
\newcommand{\saopaulo}{Universidade de S{\~a}o Paulo, Instituto de F\'{\i}sica, Caixa Postal 66318, S{\~a}o Paulo CEP05315-970, Brazil}
\newcommand{\seoulnat}{System Electronics Laboratory, Seoul National University, Seoul, South Korea}
\newcommand{\stonybrkc}{Chemistry Department, Stony Brook University, SUNY, Stony Brook, NY 11794-3400, USA}
\newcommand{\stonycrkp}{Department of Physics and Astronomy, Stony Brook University, SUNY, Stony Brook, NY 11794, USA}
\newcommand{\subatech}{SUBATECH (Ecole des Mines de Nantes, CNRS-IN2P3, Universit{\'e} de Nantes) BP 20722 - 44307, Nantes, France}
\newcommand{\tenn}{University of Tennessee, Knoxville, TN 37996, USA}
\newcommand{\titech}{Department of Physics, Tokyo Institute of Technology, Tokyo, 152-8551, Japan}
\newcommand{\tsukuba}{Institute of Physics, University of Tsukuba, Tsukuba, Ibaraki 305, Japan}
\newcommand{\vandy}{Vanderbilt University, Nashville, TN 37235, USA}
\newcommand{\waseda}{Waseda University, Advanced Research Institute for Science and Engineering, 17 Kikui-cho, Shinjuku-ku, Tokyo 162-0044, Japan}
\newcommand{\weizmann}{Weizmann Institute, Rehovot 76100, Israel}
\newcommand{\yonsei}{Yonsei University, IPAP, Seoul 120-749, Korea}
\affiliation{\abilene}
\affiliation{\acadsin}
\affiliation{\banaras}
\affiliation{\barc}
\affiliation{\bnl}
\affiliation{\caucr}
\affiliation{\ciae}
\affiliation{\cns}
\affiliation{\columbia}
\affiliation{\dapnia}
\affiliation{\debrecen}
\affiliation{\fsu}
\affiliation{\gsu}
\affiliation{\hiroshima}
\affiliation{\ihepprot}
\affiliation{\isu}
\affiliation{\jinrdubna}
\affiliation{\kaeri}
\affiliation{\kangnung}
\affiliation{\kek}
\affiliation{\kfki}
\affiliation{\korea}
\affiliation{\kurchatov}
\affiliation{\kyoto}
\affiliation{\labllr}
\affiliation{\lawllnl}
\affiliation{\losalamos}
\affiliation{\lpc}
\affiliation{\lund}
\affiliation{\muenster}
\affiliation{\myongji}
\affiliation{\nagasaki}
\affiliation{\newmex}
\affiliation{\nmsu}
\affiliation{\ornl}
\affiliation{\orsay}
\affiliation{\pnpi}
\affiliation{\riken}
\affiliation{\rikjrbrc}
\affiliation{\saispbstu}
\affiliation{\saopaulo}
\affiliation{\seoulnat}
\affiliation{\stonybrkc}
\affiliation{\stonycrkp}
\affiliation{\subatech}
\affiliation{\tenn}
\affiliation{\titech}
\affiliation{\tsukuba}
\affiliation{\vandy}
\affiliation{\waseda}
\affiliation{\weizmann}
\affiliation{\yonsei}
\author{S.S.~Adler}	\affiliation{\bnl}
\author{S.~Afanasiev}	\affiliation{\jinrdubna}
\author{C.~Aidala}	\affiliation{\bnl}
\author{N.N.~Ajitanand}	\affiliation{\stonybrkc}
\author{Y.~Akiba}	\affiliation{\kek} \affiliation{\riken}
\author{J.~Alexander}	\affiliation{\stonybrkc}
\author{R.~Amirikas}	\affiliation{\fsu}
\author{L.~Aphecetche}	\affiliation{\subatech}
\author{S.H.~Aronson}	\affiliation{\bnl}
\author{R.~Averbeck}	\affiliation{\stonycrkp}
\author{T.C.~Awes}	\affiliation{\ornl}
\author{R.~Azmoun}	\affiliation{\stonycrkp}
\author{V.~Babintsev}	\affiliation{\ihepprot}
\author{A.~Baldisseri}	\affiliation{\dapnia}
\author{K.N.~Barish}	\affiliation{\caucr}
\author{P.D.~Barnes}	\affiliation{\losalamos}
\author{B.~Bassalleck}	\affiliation{\newmex}
\author{S.~Bathe}	\affiliation{\muenster}
\author{S.~Batsouli}	\affiliation{\columbia}
\author{V.~Baublis}	\affiliation{\pnpi}
\author{A.~Bazilevsky}	\affiliation{\rikjrbrc} \affiliation{\ihepprot}
\author{S.~Belikov}	\affiliation{\isu} \affiliation{\ihepprot}
\author{Y.~Berdnikov}	\affiliation{\saispbstu}
\author{S.~Bhagavatula}	\affiliation{\isu}
\author{J.G.~Boissevain}	\affiliation{\losalamos}
\author{H.~Borel}	\affiliation{\dapnia}
\author{S.~Borenstein}	\affiliation{\labllr}
\author{M.L.~Brooks}	\affiliation{\losalamos}
\author{D.S.~Brown}	\affiliation{\nmsu}
\author{N.~Bruner}	\affiliation{\newmex}
\author{D.~Bucher}	\affiliation{\muenster}
\author{H.~Buesching}	\affiliation{\muenster}
\author{V.~Bumazhnov}	\affiliation{\ihepprot}
\author{G.~Bunce}	\affiliation{\bnl} \affiliation{\rikjrbrc}
\author{J.M.~Burward-Hoy}	\affiliation{\lawllnl} \affiliation{\stonycrkp}
\author{S.~Butsyk}	\affiliation{\stonycrkp}
\author{X.~Camard}	\affiliation{\subatech}
\author{J.-S.~Chai}	\affiliation{\kaeri}
\author{P.~Chand}	\affiliation{\barc}
\author{W.C.~Chang}	\affiliation{\acadsin}
\author{S.~Chernichenko}	\affiliation{\ihepprot}
\author{C.Y.~Chi}	\affiliation{\columbia}
\author{J.~Chiba}	\affiliation{\kek}
\author{M.~Chiu}	\affiliation{\columbia}
\author{I.J.~Choi}	\affiliation{\yonsei}
\author{J.~Choi}	\affiliation{\kangnung}
\author{R.K.~Choudhury}	\affiliation{\barc}
\author{T.~Chujo}	\affiliation{\bnl}
\author{V.~Cianciolo}	\affiliation{\ornl}
\author{Y.~Cobigo}	\affiliation{\dapnia}
\author{B.A.~Cole}	\affiliation{\columbia}
\author{P.~Constantin}	\affiliation{\isu}
\author{D.~d'Enterria}	\affiliation{\subatech}
\author{G.~David}	\affiliation{\bnl}
\author{H.~Delagrange}	\affiliation{\subatech}
\author{A.~Denisov}	\affiliation{\ihepprot}
\author{A.~Deshpande}	\affiliation{\rikjrbrc}
\author{E.J.~Desmond}	\affiliation{\bnl}
\author{A.~Devismes}	\affiliation{\stonycrkp}
\author{O.~Dietzsch}	\affiliation{\saopaulo}
\author{O.~Drapier}	\affiliation{\labllr}
\author{A.~Drees}	\affiliation{\stonycrkp}
\author{R.~du~Rietz}	\affiliation{\lund}
\author{A.~Durum}	\affiliation{\ihepprot}
\author{D.~Dutta}	\affiliation{\barc}
\author{Y.V.~Efremenko}	\affiliation{\ornl}
\author{K.~El~Chenawi}	\affiliation{\vandy}
\author{A.~Enokizono}	\affiliation{\hiroshima}
\author{H.~En'yo}	\affiliation{\riken} \affiliation{\rikjrbrc}
\author{S.~Esumi}	\affiliation{\tsukuba}
\author{L.~Ewell}	\affiliation{\bnl}
\author{D.E.~Fields}	\affiliation{\newmex} \affiliation{\rikjrbrc}
\author{F.~Fleuret}	\affiliation{\labllr}
\author{S.L.~Fokin}	\affiliation{\kurchatov}
\author{B.D.~Fox}	\affiliation{\rikjrbrc}
\author{Z.~Fraenkel}	\affiliation{\weizmann}
\author{J.E.~Frantz}	\affiliation{\columbia}
\author{A.~Franz}	\affiliation{\bnl}
\author{A.D.~Frawley}	\affiliation{\fsu}
\author{S.-Y.~Fung}	\affiliation{\caucr}
\author{S.~Garpman}   \altaffiliation{Deceased}  \affiliation{\lund}
\author{T.K.~Ghosh}	\affiliation{\vandy}
\author{A.~Glenn}	\affiliation{\tenn}
\author{G.~Gogiberidze}	\affiliation{\tenn}
\author{M.~Gonin}	\affiliation{\labllr}
\author{J.~Gosset}	\affiliation{\dapnia}
\author{Y.~Goto}	\affiliation{\rikjrbrc}
\author{R.~Granier~de~Cassagnac}	\affiliation{\labllr}
\author{N.~Grau}	\affiliation{\isu}
\author{S.V.~Greene}	\affiliation{\vandy}
\author{M.~Grosse~Perdekamp}	\affiliation{\rikjrbrc}
\author{W.~Guryn}	\affiliation{\bnl}
\author{H.-{\AA}.~Gustafsson}	\affiliation{\lund}
\author{T.~Hachiya}	\affiliation{\hiroshima}
\author{J.S.~Haggerty}	\affiliation{\bnl}
\author{H.~Hamagaki}	\affiliation{\cns}
\author{A.G.~Hansen}	\affiliation{\losalamos}
\author{E.P.~Hartouni}	\affiliation{\lawllnl}
\author{M.~Harvey}	\affiliation{\bnl}
\author{R.~Hayano}	\affiliation{\cns}
\author{N.~Hayashi}	\affiliation{\riken}
\author{X.~He}	\affiliation{\gsu}
\author{M.~Heffner}	\affiliation{\lawllnl}
\author{T.K.~Hemmick}	\affiliation{\stonycrkp}
\author{J.M.~Heuser}	\affiliation{\stonycrkp}
\author{M.~Hibino}	\affiliation{\waseda}
\author{J.C.~Hill}	\affiliation{\isu}
\author{W.~Holzmann}	\affiliation{\stonybrkc}
\author{K.~Homma}	\affiliation{\hiroshima}
\author{B.~Hong}	\affiliation{\korea}
\author{A.~Hoover}	\affiliation{\nmsu}
\author{T.~Ichihara}	\affiliation{\riken} \affiliation{\rikjrbrc}
\author{V.V.~Ikonnikov}	\affiliation{\kurchatov}
\author{K.~Imai}	\affiliation{\kyoto} \affiliation{\riken}
\author{D.~Isenhower}	\affiliation{\abilene}
\author{M.~Ishihara}	\affiliation{\riken}
\author{M.~Issah}	\affiliation{\stonybrkc}
\author{A.~Isupov}	\affiliation{\jinrdubna}
\author{B.V.~Jacak}	\affiliation{\stonycrkp}
\author{W.Y.~Jang}	\affiliation{\korea}
\author{Y.~Jeong}	\affiliation{\kangnung}
\author{J.~Jia}	\affiliation{\stonycrkp}
\author{O.~Jinnouchi}	\affiliation{\riken}
\author{B.M.~Johnson}	\affiliation{\bnl}
\author{S.C.~Johnson}	\affiliation{\lawllnl}
\author{K.S.~Joo}	\affiliation{\myongji}
\author{D.~Jouan}	\affiliation{\orsay}
\author{S.~Kametani}	\affiliation{\cns} \affiliation{\waseda}
\author{N.~Kamihara}	\affiliation{\titech} \affiliation{\riken}
\author{J.H.~Kang}	\affiliation{\yonsei}
\author{S.S.~Kapoor}	\affiliation{\barc}
\author{K.~Katou}	\affiliation{\waseda}
\author{S.~Kelly}	\affiliation{\columbia}
\author{B.~Khachaturov}	\affiliation{\weizmann}
\author{A.~Khanzadeev}	\affiliation{\pnpi}
\author{J.~Kikuchi}	\affiliation{\waseda}
\author{D.H.~Kim}	\affiliation{\myongji}
\author{D.J.~Kim}	\affiliation{\yonsei}
\author{D.W.~Kim}	\affiliation{\kangnung}
\author{E.~Kim}	\affiliation{\seoulnat}
\author{G.-B.~Kim}	\affiliation{\labllr}
\author{H.J.~Kim}	\affiliation{\yonsei}
\author{E.~Kistenev}	\affiliation{\bnl}
\author{A.~Kiyomichi}	\affiliation{\tsukuba}
\author{K.~Kiyoyama}	\affiliation{\nagasaki}
\author{C.~Klein-Boesing}	\affiliation{\muenster}
\author{H.~Kobayashi}	\affiliation{\riken} \affiliation{\rikjrbrc}
\author{L.~Kochenda}	\affiliation{\pnpi}
\author{V.~Kochetkov}	\affiliation{\ihepprot}
\author{D.~Koehler}	\affiliation{\newmex}
\author{T.~Kohama}	\affiliation{\hiroshima}
\author{M.~Kopytine}	\affiliation{\stonycrkp}
\author{D.~Kotchetkov}	\affiliation{\caucr}
\author{A.~Kozlov}	\affiliation{\weizmann}
\author{P.J.~Kroon}	\affiliation{\bnl}
\author{C.H.~Kuberg}	\affiliation{\abilene} \affiliation{\losalamos}
\author{K.~Kurita}	\affiliation{\rikjrbrc}
\author{Y.~Kuroki}	\affiliation{\tsukuba}
\author{M.J.~Kweon}	\affiliation{\korea}
\author{Y.~Kwon}	\affiliation{\yonsei}
\author{G.S.~Kyle}	\affiliation{\nmsu}
\author{R.~Lacey}	\affiliation{\stonybrkc}
\author{V.~Ladygin}	\affiliation{\jinrdubna}
\author{J.G.~Lajoie}	\affiliation{\isu}
\author{A.~Lebedev}	\affiliation{\isu} \affiliation{\kurchatov}
\author{S.~Leckey}	\affiliation{\stonycrkp}
\author{D.M.~Lee}	\affiliation{\losalamos}
\author{S.~Lee}	\affiliation{\kangnung}
\author{M.J.~Leitch}	\affiliation{\losalamos}
\author{X.H.~Li}	\affiliation{\caucr}
\author{H.~Lim}	\affiliation{\seoulnat}
\author{A.~Litvinenko}	\affiliation{\jinrdubna}
\author{M.X.~Liu}	\affiliation{\losalamos}
\author{Y.~Liu}	\affiliation{\orsay}
\author{C.F.~Maguire}	\affiliation{\vandy}
\author{Y.I.~Makdisi}	\affiliation{\bnl}
\author{A.~Malakhov}	\affiliation{\jinrdubna}
\author{V.I.~Manko}	\affiliation{\kurchatov}
\author{Y.~Mao}	\affiliation{\ciae} \affiliation{\riken}
\author{G.~Martinez}	\affiliation{\subatech}
\author{M.D.~Marx}	\affiliation{\stonycrkp}
\author{H.~Masui}	\affiliation{\tsukuba}
\author{F.~Matathias}	\affiliation{\stonycrkp}
\author{T.~Matsumoto}	\affiliation{\cns} \affiliation{\waseda}
\author{P.L.~McGaughey}	\affiliation{\losalamos}
\author{E.~Melnikov}	\affiliation{\ihepprot}
\author{F.~Messer}	\affiliation{\stonycrkp}
\author{Y.~Miake}	\affiliation{\tsukuba}
\author{J.~Milan}	\affiliation{\stonybrkc}
\author{T.E.~Miller}	\affiliation{\vandy}
\author{A.~Milov}	\affiliation{\stonycrkp} \affiliation{\weizmann}
\author{S.~Mioduszewski}	\affiliation{\bnl}
\author{R.E.~Mischke}	\affiliation{\losalamos}
\author{G.C.~Mishra}	\affiliation{\gsu}
\author{J.T.~Mitchell}	\affiliation{\bnl}
\author{A.K.~Mohanty}	\affiliation{\barc}
\author{D.P.~Morrison}	\affiliation{\bnl}
\author{J.M.~Moss}	\affiliation{\losalamos}
\author{F.~M{\"u}hlbacher}	\affiliation{\stonycrkp}
\author{D.~Mukhopadhyay}	\affiliation{\weizmann}
\author{M.~Muniruzzaman}	\affiliation{\caucr}
\author{J.~Murata}	\affiliation{\riken} \affiliation{\rikjrbrc}
\author{S.~Nagamiya}	\affiliation{\kek}
\author{J.L.~Nagle}	\affiliation{\columbia}
\author{T.~Nakamura}	\affiliation{\hiroshima}
\author{B.K.~Nandi}	\affiliation{\caucr}
\author{M.~Nara}	\affiliation{\tsukuba}
\author{J.~Newby}	\affiliation{\tenn}
\author{P.~Nilsson}	\affiliation{\lund}
\author{A.S.~Nyanin}	\affiliation{\kurchatov}
\author{J.~Nystrand}	\affiliation{\lund}
\author{E.~O'Brien}	\affiliation{\bnl}
\author{C.A.~Ogilvie}	\affiliation{\isu}
\author{H.~Ohnishi}	\affiliation{\bnl} \affiliation{\riken}
\author{I.D.~Ojha}	\affiliation{\vandy} \affiliation{\banaras}
\author{K.~Okada}	\affiliation{\riken}
\author{M.~Ono}	\affiliation{\tsukuba}
\author{V.~Onuchin}	\affiliation{\ihepprot}
\author{A.~Oskarsson}	\affiliation{\lund}
\author{I.~Otterlund}	\affiliation{\lund}
\author{K.~Oyama}	\affiliation{\cns}
\author{K.~Ozawa}	\affiliation{\cns}
\author{D.~Pal}	\affiliation{\weizmann}
\author{A.P.T.~Palounek}	\affiliation{\losalamos}
\author{V.~Pantuev}	\affiliation{\stonycrkp}
\author{V.~Papavassiliou}	\affiliation{\nmsu}
\author{J.~Park}	\affiliation{\seoulnat}
\author{A.~Parmar}	\affiliation{\newmex}
\author{S.F.~Pate}	\affiliation{\nmsu}
\author{T.~Peitzmann}	\affiliation{\muenster}
\author{J.-C.~Peng}	\affiliation{\losalamos}
\author{V.~Peresedov}	\affiliation{\jinrdubna}
\author{C.~Pinkenburg}	\affiliation{\bnl}
\author{R.P.~Pisani}	\affiliation{\bnl}
\author{F.~Plasil}	\affiliation{\ornl}
\author{M.L.~Purschke}	\affiliation{\bnl}
\author{A.K.~Purwar}	\affiliation{\stonycrkp}
\author{J.~Rak}	\affiliation{\isu}
\author{I.~Ravinovich}	\affiliation{\weizmann}
\author{K.F.~Read}	\affiliation{\ornl} \affiliation{\tenn}
\author{M.~Reuter}	\affiliation{\stonycrkp}
\author{K.~Reygers}	\affiliation{\muenster}
\author{V.~Riabov}	\affiliation{\pnpi} \affiliation{\saispbstu}
\author{Y.~Riabov}	\affiliation{\pnpi}
\author{G.~Roche}	\affiliation{\lpc}
\author{A.~Romana}	\affiliation{\labllr}
\author{M.~Rosati}	\affiliation{\isu}
\author{P.~Rosnet}	\affiliation{\lpc}
\author{S.S.~Ryu}	\affiliation{\yonsei}
\author{M.E.~Sadler}	\affiliation{\abilene}
\author{N.~Saito}	\affiliation{\riken} \affiliation{\rikjrbrc}
\author{T.~Sakaguchi}	\affiliation{\cns} \affiliation{\waseda}
\author{M.~Sakai}	\affiliation{\nagasaki}
\author{S.~Sakai}	\affiliation{\tsukuba}
\author{V.~Samsonov}	\affiliation{\pnpi}
\author{L.~Sanfratello}	\affiliation{\newmex}
\author{R.~Santo}	\affiliation{\muenster}
\author{H.D.~Sato}	\affiliation{\kyoto} \affiliation{\riken}
\author{S.~Sato}	\affiliation{\bnl} \affiliation{\tsukuba}
\author{S.~Sawada}	\affiliation{\kek}
\author{Y.~Schutz}	\affiliation{\subatech}
\author{V.~Semenov}	\affiliation{\ihepprot}
\author{R.~Seto}	\affiliation{\caucr}
\author{M.R.~Shaw}	\affiliation{\abilene} \affiliation{\losalamos}
\author{T.K.~Shea}	\affiliation{\bnl}
\author{T.-A.~Shibata}	\affiliation{\titech} \affiliation{\riken}
\author{K.~Shigaki}	\affiliation{\hiroshima} \affiliation{\kek}
\author{T.~Shiina}	\affiliation{\losalamos}
\author{C.L.~Silva}	\affiliation{\saopaulo}
\author{D.~Silvermyr}	\affiliation{\losalamos} \affiliation{\lund}
\author{K.S.~Sim}	\affiliation{\korea}
\author{C.P.~Singh}	\affiliation{\banaras}
\author{V.~Singh}	\affiliation{\banaras}
\author{M.~Sivertz}	\affiliation{\bnl}
\author{A.~Soldatov}	\affiliation{\ihepprot}
\author{R.A.~Soltz}	\affiliation{\lawllnl}
\author{W.E.~Sondheim}	\affiliation{\losalamos}
\author{S.P.~Sorensen}	\affiliation{\tenn}
\author{I.V.~Sourikova}	\affiliation{\bnl}
\author{F.~Staley}	\affiliation{\dapnia}
\author{P.W.~Stankus}	\affiliation{\ornl}
\author{E.~Stenlund}	\affiliation{\lund}
\author{M.~Stepanov}	\affiliation{\nmsu}
\author{A.~Ster}	\affiliation{\kfki}
\author{S.P.~Stoll}	\affiliation{\bnl}
\author{T.~Sugitate}	\affiliation{\hiroshima}
\author{J.P.~Sullivan}	\affiliation{\losalamos}
\author{E.M.~Takagui}	\affiliation{\saopaulo}
\author{A.~Taketani}	\affiliation{\riken} \affiliation{\rikjrbrc}
\author{M.~Tamai}	\affiliation{\waseda}
\author{K.H.~Tanaka}	\affiliation{\kek}
\author{Y.~Tanaka}	\affiliation{\nagasaki}
\author{K.~Tanida}	\affiliation{\riken}
\author{M.J.~Tannenbaum}	\affiliation{\bnl}
\author{P.~Tarj{\'a}n}	\affiliation{\debrecen}
\author{J.D.~Tepe}	\affiliation{\abilene} \affiliation{\losalamos}
\author{T.L.~Thomas}	\affiliation{\newmex}
\author{J.~Tojo}	\affiliation{\kyoto} \affiliation{\riken}
\author{H.~Torii}	\affiliation{\kyoto} \affiliation{\riken}
\author{R.S.~Towell}	\affiliation{\abilene}
\author{I.~Tserruya}	\affiliation{\weizmann}
\author{H.~Tsuruoka}	\affiliation{\tsukuba}
\author{S.K.~Tuli}	\affiliation{\banaras}
\author{H.~Tydesj{\"o}}	\affiliation{\lund}
\author{N.~Tyurin}	\affiliation{\ihepprot}
\author{H.W.~van~Hecke}	\affiliation{\losalamos}
\author{J.~Velkovska}	\affiliation{\bnl} \affiliation{\stonycrkp}
\author{M.~Velkovsky}	\affiliation{\stonycrkp}
\author{V.~Veszpr{\'e}mi}	\affiliation{\debrecen}
\author{L.~Villatte}	\affiliation{\tenn}
\author{A.A.~Vinogradov}	\affiliation{\kurchatov}
\author{M.A.~Volkov}	\affiliation{\kurchatov}
\author{E.~Vznuzdaev}	\affiliation{\pnpi}
\author{X.R.~Wang}	\affiliation{\gsu}
\author{Y.~Watanabe}	\affiliation{\riken} \affiliation{\rikjrbrc}
\author{S.N.~White}	\affiliation{\bnl}
\author{F.K.~Wohn}	\affiliation{\isu}
\author{C.L.~Woody}	\affiliation{\bnl}
\author{W.~Xie}	\affiliation{\caucr}
\author{Y.~Yang}	\affiliation{\ciae}
\author{A.~Yanovich}	\affiliation{\ihepprot}
\author{S.~Yokkaichi}	\affiliation{\riken} \affiliation{\rikjrbrc}
\author{G.R.~Young}	\affiliation{\ornl}
\author{I.E.~Yushmanov}	\affiliation{\kurchatov}
\author{W.A.~Zajc}\email[PHENIX Spokesperson:]{zajc@nevis.columbia.edu}	\affiliation{\columbia}
\author{C.~Zhang}	\affiliation{\columbia}
\author{S.~Zhou}	\affiliation{\ciae}
\author{S.J.~Zhou}	\affiliation{\weizmann}
\author{L.~Zolin}	\affiliation{\jinrdubna}
\collaboration{PHENIX Collaboration} \noaffiliation

\date{\today}

\begin{abstract}
The invariant differential cross section for inclusive electron production in
$p + p$ collisions at $\sqrt{s} = 200$~GeV has been measured by the PHENIX 
experiment at the Relativistic Heavy Ion Collider over the transverse momentum
range $0.4 \le p_T \le 5.0$~GeV/$c$ at midrapidity ($|\eta| \le 0.35$). 
The contribution to the inclusive electron spectrum from semileptonic decays 
of hadrons carrying heavy flavor, {\it i.e.} charm quarks or, at high $p_T$, 
bottom quarks, is determined via three independent methods. 
The resulting electron spectrum from heavy flavor decays is compared to 
recent leading and next-to-leading order perturbative QCD calculations. 
The total cross section of charm quark-antiquark pair production is determined 
as 
$\sigma_{c\bar{c}} = 0.92 \pm 0.15 {\rm (stat.)} \pm 0.54 {\rm (sys.)}$~mb.
\end{abstract}

% insert suggested PACS numbers in braces on next line
\pacs{13.85.Qk, 13.20.Fc, 13.20.He, 25.75.Dw} 

%\maketitle must follow title, authors, abstract, \pacs, and \keywords
\maketitle

The production of hadrons carrying heavy quarks, {\it i.e.} charm or bottom, 
serves as a crucial proving ground for quantum chromodynamics (QCD), the 
theory of the strong interaction. 
Because of the large quark masses, charm and bottom production can be treated 
by perturbative QCD (pQCD) even at small momenta without being significantly
affected by additional soft processes~\cite{mangano93}. 
This is in distinct contrast to the production of particles composed solely of
light quarks, which can be evaluated perturbatively only for sufficiently 
large momenta.
Consequently, pQCD calculations of heavy quark production are expected to be 
reliable over the full momentum range experimentally accessible at collider 
energies.

For bottom production, next-to-leading order (NLO) calculations are in 
reasonable agreement with data~\cite{cacciari04mangano04}.
Charm measurements at $\sqrt{s} = 1.96$~TeV exist for high transverse momentum 
($p_T$) only~\cite{acosta03}, where the cross section is higher than NLO 
predictions by $\ge 50$\%.
However, these discrepancies are within the substantial experimental and 
theoretical uncertainties~\cite{acosta03}.
At the Relativistic Heavy Ion Collider (RHIC), charm data have been shown for
$p + p$  and $d + Au$ collisions at 
$\sqrt{s_{NN}} = 200$~GeV~\cite{star_dau,phenix_dau} as well as for $Au + Au$ 
collisions at 130 and 200~GeV~\cite{phenix_auau130,phenix_auau200}.
Further measurements are crucial for a better understanding of heavy flavor 
production at RHIC. 
In particular, the relevance of higher order processes and other production 
mechanisms like jet fragmentation is unclear.

We report on the midrapidity production ($|\eta| \le 0.35$) of inclusive 
electrons, $(e^+ + e^-)/2$, in $p + p$ collisions at $\sqrt{s} = 200$~GeV 
measured by the PHENIX experiment~\cite{phenix_nim} at RHIC. 
Contributions from semileptonic heavy flavor decays are extracted in the 
electron $p_T$ range $0.4 \le p_T \le 5.0$~GeV/$c$. 
The resulting invariant differential cross section is an important benchmark 
for pQCD calculations of heavy quark production. 
Furthermore, it provides a crucial baseline for measurements in nuclear 
collisions at RHIC.
Since hadronic heavy flavor production is expected to be dominated by 
initial parton scattering, systematic studies in $p + p$ and $d + Au$ 
collisions should be sensitive to the nucleon parton distribution functions 
as well as to nuclear modifications of these such as shadowing~\cite{lin96}. 
In $Au + Au$ collisions, heavy quarks constitute a unique and, with the data 
presented here, calibrated probe for the created hot and dense medium. 
Possible medium effects on heavy flavor probes include energy 
loss~\cite{dokshitzer01,armesto05}, azimuthal asymmetry~\cite{lin03greco04}, 
and quarkonia suppression~\cite{matsui86} or enhancement~\cite{pbm,thews}.

The data used here were recorded by PHENIX during RHIC Run-2. 
Beam-beam counters (BBC), positioned at pseudorapidities 
$3.1 < |\eta| < 3.9$, measured the collision vertex and provided the 
minimum bias (MB) interaction trigger defined by at least one hit on each side 
of the vertex. 
Events containing high $p_T$ electrons were selected by an additional level-1 
trigger in coincidence with the MB trigger. 
This level-1 trigger required a minimum energy deposit of 0.75~GeV in a 
$2 \times 2$ tile of towers in the electromagnetic calorimeter 
(EMC)~\cite{phenix_pi0pp}. 
After a vertex cut of $|z_{vtx}| < 20$~cm, an equivalent of 
$465 \times 10^6$ MB events sampled by the EMC trigger was analyzed in 
addition to the $15 \times 10^6$ events recorded with the MB trigger itself.

The PHENIX east arm spectrometer ($|\eta| < 0.35$, $\Delta \phi = \pi/2$)
includes a drift chamber and a pad chamber layer for charged particle 
tracking. 
Tracks were confirmed by hits in the EMC matching in position with the 
track projection within $3\sigma$. 
Electron candidates required at least two associated hits in the ring imaging 
\v{C}erenkov detector (RICH) in the projected ring area. 
Random coincidences of hadron tracks and hits in the RICH occured with a
probability of $(3.0\pm1.5) \times 10^{-4}$.
For electrons the energy $E$ deposited in the EMC equals the momentum $p$.
Requiring $|(E-p)/p| < 3\sigma$, a total charged hadron rejection factor of 
about 10$^4$ (10$^5$) was achieved for $p_T = 0.4$ $(\ge 2.0)$~GeV/$c$.
Remaining background ($<1$~\%) was measured via event mixing and subtracted 
statistically.

The differential cross section for electron production was calculated as
\begin{equation}
E \frac{d^3\sigma}{dp^3} = \frac{1}{\epsilon_{bias} \int{{\cal{L}}dt}}
			   \frac{N_e}{2\pi p_T \Delta y \Delta p_T}
			   \frac{1}{A \epsilon_{rec}},
\end{equation}
where $\int{{\cal{L}}dt}$ is the integrated luminosity measured with the MB 
trigger or sampled with the EMC trigger, respectively, $\epsilon_{bias}$
is the probability for an electron event to fulfill the MB trigger condition, 
$N_e$ is the measured electron yield, and $A \epsilon_{rec}$ is the product of 
geometrical acceptance and reconstruction efficiency. 
For the EMC triggered sample, $\epsilon_{rec}$ includes the trigger efficiency
$\epsilon_{lvl1}$.

$\int{{\cal{L}}dt}$ is calculated as $N_{MB}/\sigma_{BBC}$, where $N_{MB}$ 
is the number of MB triggers or, for the EMC triggered sample, the number 
of EMC triggers divided by the measured fraction of MB events which 
simultaneously fulfill the EMC trigger criterion.
With the MB trigger cross section 
$\sigma_{BBC} = 21.8 \pm 2.1$~mb~\cite{phenix_pi0pp}, the analyzed data 
samples correspond to integrated luminosities of 0.7 nb$^{-1}$ (MB trigger)
and 21 nb$^{-1}$ (EMC trigger), respectively.
The $p_T$ independent trigger bias $\epsilon_{bias} = 0.75 \pm 0.02$ was 
measured for events containing a $\pi^0$ with 
$p_T > 1.5$~GeV/$c$~\cite{phenix_pi0pp} and confirmed for charged hadrons with
$p_T > 0.2$~GeV/$c$~\cite{phenix_ppg050}, indicating a universal bias both for 
hard and soft processes.
$A \epsilon_{rec}$ was calculated as a function of $p_T$ ($< 10$~\% variation 
over the full $p_T$ range) in a GEANT~\cite{geant} simulation of electrons 
with flat distributions in rapidity ($|y| < 0.6$), azimuth ($0 < \phi < 2\pi$),
and event vertex ($|z| < 30$~cm) as input. 
The simulated detector response was carefully tuned to match the real detector.
Rigorous fiducial cuts were applied to eliminate active area mismatches 
between data and simulation as well as run-by-run variations.
The trigger efficiency $\epsilon_{lvl1}$, evaluated for single electrons
in the fiducial area, rises from zero at low $p_T$ to $95 \pm 5$\% for 
$p_T > 2$~GeV/$c$.
Finally the effect of finite bin width in $p_T$ was appropriately corrected 
for.

\begin{figure}[t]
\includegraphics[width=1.0\linewidth]{Fig_1.eps}
\caption{\label{fig:PPG037_Fig1} (Color online) Inclusive electron invariant
differential cross section, measured in $p + p$ collisions at 
$\sqrt{s} = 200$~GeV, compared with all contributions from electron sources 
included in the background {\it cocktail} (upper panel). Error bars (boxes) 
correspond to statistical (systematic) uncertainties. Relative contributions
of all electron sources to the background {\it cocktail} (lower panel).}
\end{figure}

The corrected electron spectra from the MB and EMC triggered samples cover 
$p_T$ ranges of $0.4 < p_T < 2.0$~GeV/$c$ and $0.6 < p_T < 5.0$~GeV/$c$, 
respectively. 
They are consistent with each other within the statistical uncertainties 
in the $p_T$ region of overlap. 
The weighted average of both measurements is shown in 
Fig.~\ref{fig:PPG037_Fig1}. 

The systematic uncertainty of the inclusive electron spectrum is about 12\%, 
almost $p_T$ independent, calculated as the sum in quadrature of contributions 
from the acceptance calculation (7\%), electron identification cuts (5.2\%), 
run-by-run variations (4\%), tracking efficiency (3\%), momentum scale 
(1 - 5\%), and other smaller uncertainties. 
The value of 12\% does not include the 9.6\% uncertainty of the absolute 
normalization.

The invariant cross section of electrons from heavy flavor decays was 
determined by subtracting a {\it cocktail} of contributions from other 
sources from the inclusive data.
The most important background is the $\pi^0$ Dalitz decay which was 
calculated with a hadron decay generator using a parameterization of measured 
$\pi^0$~\cite{phenix_pi0pp} and $\pi^\pm$~\cite{phenix_pichargepp} spectra 
as input. 
The spectral shapes of other light hadrons $h$ were obtained from the pion 
spectra by $m_T$ scaling. 
Within this approach the ratios $h/\pi^0$ are constant at high $p_T$ and for 
the relative normalization we used: 
$\eta/\pi^0 = 0.45 \pm 0.10$~\cite{phenix_ppg051}, $\rho/\pi^0 = 1.0 \pm 0.3$, 
$\omega/\pi^0 = 1.0 \pm 0.3$, $\eta'/\pi^0 = 0.25 \pm 0.08$, and 
$\phi/\pi^0 = 0.40 \pm 0.12$. 
Only the $\eta$ contribution is of any practical relevance. 
Another major electron source is the conversion of photons, mainly from 
$\pi^0 \rightarrow \gamma\gamma$ decays, in material in the acceptance. 
The spectra of electrons from conversions and Dalitz decays are very 
similar.
In a GEANT simulation of $\pi^0$ decays, the ratio of electrons from 
conversions to electrons from Dalitz decays was determined as 
$0.73 \pm 0.07$, essentially $p_T$ independent. 
Contributions from photon conversions from other sources were taken 
into account as well. 
In addition, electrons from kaon decays ($K_{e3}$), determined in a GEANT 
simulation based on measured kaon spectra~\cite{phenix_pichargepp}, and 
electrons from external as well as internal conversions of direct 
photons~\cite{vogelsang,phenix_directgammapp} were considered in the cocktail.
All background sources are compared with the inclusive data in the upper panel 
of Fig.~\ref{fig:PPG037_Fig1} with the relative contributions shown in the 
lower panel. 
The total systematic uncertainty of the cocktail is about 12\%, essentially 
$p_T$ independent. 
This uncertainty is dominated by the systematic error of the pion 
parameterization ($\approx 10$\%). 
Other systematic uncertainties, mainly the $\eta/\pi^0$ normalization and, 
at high $p_T$, the contribution from direct radiation, are much smaller.

\begin{figure}[t]
\includegraphics[width=1.0\linewidth]{Fig_2.eps}
\caption {\label{fig:PPG037_Fig2} Ratio of electrons from heavy flavor decays 
(non-photonic) and other sources (photonic) , $R_{NP}$, for three 
independent analysis methods. Error bars (boxes) are statistical 
({\it cocktail} systematic) uncertainties.}
\end{figure}

Given the small amount of material in the acceptance 
(Be beam pipe: 0.29~\% $X_0$; air: 0.28~\% $X_0$) the ratio $R_{NP}$ of 
non-photonic electrons from heavy flavor decays to background from photonic 
sources is large ($R_{NP} > 1$ for $p_T > 1.5$~GeV/$c$) as shown in
Fig.~\ref{fig:PPG037_Fig2}.
Two complementary analysis methods confirm the {\it cocktail} result:

The {\it converter} technique~\cite{phenix_auau200} compares 
electron spectra measured with an additional photon converter 
$X_{C} = 1.67$~\% $X_0$ introduced into the acceptance to measurements 
without converter. 
The converter increases the contribution from conversions and Dalitz 
decays by a fixed factor, which was determined precisely via GEANT 
simulations. 
Thus, the electron spectra from photonic and non-photonic sources can be 
deduced (Fig.~\ref{fig:PPG037_Fig2}). 
The drawbacks of the {\it converter} method are the limitation in statistics 
of the converter run period and the fact that the photonic contribution is 
small at high $p_T$.

The {\it $e\gamma$ coincidence} technique evaluates the correlation of
electrons and photons via their invariant mass. 
Electrons from $\pi^0$ Dalitz decays or the conversion of one of the photons
from $\pi^0 \rightarrow \gamma\gamma$ decays are correlated with a photon, in 
contrast to electrons from semileptonic heavy flavor decays.
Comparing the measured $e\gamma$ coincidence rate with the simulated rate
for single $\pi^0$ events, allows to deduce $R_{NP}$ as shown in 
Fig.~\ref{fig:PPG037_Fig2}, once corrections for contributions from other 
photonic sources are applied. 

After subtracting the background cocktail from the inclusive electron 
spectrum the invariant differential cross section of electrons from 
heavy flavor decays is shown in Fig.~\ref{fig:PPG037_Fig3} compared with 
two theoretical predictions. 
A leading order (LO) PYTHIA calculation, tuned to existing charm and bottom 
hadroproduction measurements~\cite{pythia_tuned}, is in reasonable agreement 
with the data for $p_T < 1.5$~GeV/$c$, but underestimates the cross section 
at higher $p_T$. 
It is important to note that this calculation includes a scale factor $K = 3.5$
to accomodate for neglected NLO contributions. 
A {\it Fixed-Order plus Next-to-Leading-Log} (FONLL) pQCD 
calculation~\cite{cacciari05} still leaves room for further contributions 
beyond the included NLO processes.
The predicted contribution from bottom decays is irrelevant for the
electron cross section at $p_T < 3$~GeV/$c$ and becomes significant only for 
$p_T > 4$~GeV/$c$.

\begin{figure}[t]
\includegraphics[width=1.0\linewidth]{Fig_3.eps}
\caption {\label{fig:PPG037_Fig3} Invariant differential cross section of 
electrons from heavy flavor decays compared with PYTHIA LO (with $K = 3.5$) 
and FONLL pQCD calculations (upper panel). Error bars (brackets) show 
statistical (systematic) uncertainties. For the FONLL calculation contributions
from charm and bottom decays are shown separately. Ratio of data and FONLL
calculation (lower panel) with experimental statistical (error bars) and
systematic (brackets) uncertainties as well as the theoretical uncertainty
(grey band). The solid line corresponds to the ratio of PYTHIA and FONLL.}
\end{figure}

The charm production cross section was derived from the integrated electron
cross section for $p_{T} > p_{T,low} = 0.6 (0.8)$~GeV/$c$
($d\sigma_{e}^{p_{T,low}}/dy = 4.78 (2.15) \pm 0.78 (0.46) {\rm (stat.)} 
\pm 1.74 (0.68) {\rm (sys.)}$~$\times 10^{-3}$~mb).
Since in the low $p_T$ region, which dominates the total cross section, PYTHIA
describes the measured spectrum reasonably well, the total charm cross section 
was determined by extrapolating the properly scaled PYTHIA spectrum to 
$p_T = 0$~GeV/$c$. 
First the PYTHIA spectra for electrons from charm and bottom decays were fit 
to the data for $p_T > 0.6$~GeV/$c$, with only the normalizations as free 
parameters. 
The resulting midrapidity charm production cross section was determined as 
$d\sigma_{c\bar{c}}/dy = 0.20 \pm 0.03 {\rm (stat.)} \pm 0.11 {\rm (sys.)}$~mb,
where the systematic error is dominated by the uncertainty of the electron 
spectrum itself ($\approx$56\%), evaluated by refitting PYTHIA to the
data at the minimum and maximum of the 1$\sigma$ systematic error band. 
Additional uncertainties from the relative ratios of different charmed hadron 
species and their branching ratios into electrons ($\approx$9\%) and the 
variation of the PYTHIA spectral shape ($\approx$11\%)~\cite{phenix_auau200} 
were added in quadrature.
The rapidity integrated cross section was determined as
$\sigma_{c\bar{c}} = 0.92 \pm 0.15 {\rm (stat.)} \pm 0.54 {\rm (sys.)}$~mb,
where various parton distribution functions (GRV98LO and MRST(c-g)~\cite{pdf} 
in addition to the default CTEQ5L~\cite{cteq5l}) were used for the 
extrapolation, with an associated extra systematic error of 
$\approx$6\%~\cite{phenix_auau200} added in quadrature.

Within errors the integrated charm cross section is compatible with data from 
$Au + Au$ collisions~\cite{phenix_auau200} (minimum bias value: 
$0.622 \pm 0.057 \pm 0.160$~mb per $NN$ collision) and from $d + Au$ 
collisions~\cite{star_dau} ($1.3 \pm 0.2 \pm 0.4$~mb) at the same 
$\sqrt{s_{NN}} = 200$~GeV. 
The FONLL cross section is smaller 
($\sigma_{c\bar{c}}^{FONLL} = 0.256^{+0.400}_{-0.146}$~mb) but it is still 
compatible with the data. 
Our measurement does not allow to deduce a bottom cross section, which is 
predicted by FONLL as 
$\sigma_{b\bar{b}}^{FONLL} = 1.87^{+0.99}_{-0.67}$~$\mu$b.

In conclusion, we have measured single electrons from heavy flavor decays
in $p + p$ collisions at $\sqrt{s} = 200$~GeV. 
These data provide a crucial benchmark for pQCD heavy quark calculations. 
We observe that above $p_T \approx 2$~GeV/$c$ the electron spectrum is 
significantly harder than predicted by a LO PYTHIA charm and bottom 
calculation. 
Contributions to the charm production cross section in excess of the considered
FONLL calculation, {\it e.g.} from jet fragmentation, can not be excluded.
The new data reported here provide an important baseline
for the study of medium effects on heavy quark production at RHIC.

%\section{Acknowledgements}   % Run-2 short form for PRL

We thank the staff of the Collider-Accelerator and Physics
Departments at BNL for their vital contributions.  We acknowledge
support from the Department of Energy and NSF (U.S.A.), 
MEXT and JSPS (Japan), CNPq and FAPESP (Brazil), NSFC (China), 
CNRS-IN2P3 and CEA (France), 
BMBF, DAAD, and AvH (Germany), 
OTKA (Hungary), DAE and DST (India), ISF (Israel), 
KRF and CHEP (Korea), RMIST, RAS, and RMAE (Russia), 
VR and KAW (Sweden), U.S. CRDF for the FSU, 
US-Hungarian NSF-OTKA-MTA, and US-Israel BSF.

\def\IJMPA{{Int. J. Mod. Phys.}~{\bf A}}
\def\JPG{{J. Phys}~{\bf G}}
\def\NCA{Nuovo Cimento}
\def\NIM{Nucl. Instrum. Methods}
\def\NIMA{{Nucl. Instrum. Methods}~{\bf A}}
\def\NPA{{Nucl. Phys.}~{\bf A}}
\def\NPB{{Nucl. Phys.}~{\bf B}}
\def\PLB{Phys. Lett. B}
\def\PLC{Phys. Repts.\ }
\def\PRL{Phys. Rev. Lett.\ }
\def\PRD{Phys. Rev. D}
\def\PRC{Phys. Rev. C}
\def\ZPC{{Z. Phys.}~{\bf C}}
\def\etal{{\it et al.} }

\begin{references}
\bibitem{mangano93} 
  M.L. Mangano \etal, \NPB {\bf 405}, 507 (1993).
\bibitem{cacciari04mangano04}
  M. Cacciari, hep-ph/0407187; 
  M.L. Mangano, hep-ph/0411020.
\bibitem{acosta03}
  D. Acosta \etal, \PRL {\bf 91}, 241804 (2003).
\bibitem{star_dau}
  J. Adams \etal, \PRL {\bf 94} 062301 (2005).
\bibitem{phenix_dau}
  S. Kelly \etal, \JPG {\bf 30}, S1189 (2004).
\bibitem{phenix_auau130}
  K. Adcox \etal, \PRL {\bf 88}, 192303 (2002).
\bibitem{phenix_auau200}
  S.S. Adler \etal, \PRL {\bf 94}, 082301 (2005).
\bibitem{phenix_nim}
  K. Adcox \etal, \NIMA {\bf 499}, 469 (2003).  
\bibitem{lin96}
  Z. Lin and M. Gyulassy, \PRL {\bf 77}, 1222 (1996).
\bibitem{dokshitzer01}
  Y.L. Dokshitzer and D.E. Kharzeev, \PLB {\bf 519}, 199 (2001).
\bibitem{armesto05}
  N. Armesto, A. Dainese, C.A. Salgado, and U.A. Wiedemann
  \PRD {\bf 71}, 054027 (2005).
\bibitem{lin03greco04}
  Z.W. Lin and D. Molnar, \PRC {\bf 68}, 044901 (2003);
  V. Greco, C.M. Ko, and R. Rapp, \PLB {\bf 595}, 202 (2004).
\bibitem{matsui86}
  T. Matsui and H. Satz, \PLB {\bf 178}, 416 (1986).
\bibitem{pbm}
  P. Braun-Munzinger and J. Stachel, \PLB {\bf 490}, 196 (2000).
\bibitem{thews}
  R.L. Thews, M. Schroedter, and J. Rafelski, \PRC {\bf 63}, 054905 (2001).
\bibitem{phenix_pi0pp}
  S.S. Adler \etal, \PRL {\bf 91}, 241803 (2003).
\bibitem{phenix_ppg050}
  S.S. Adler \etal, submitted for publication in \PRL, hep-ex/0507073.
\bibitem{geant}
  GEANT 3.21, CERN program library.
\bibitem{phenix_pichargepp}
  F. Matathias \etal, \JPG {\bf 30}, S1113 (2004).
\bibitem{phenix_ppg051}
  S.S. Adler \etal, in preparation.
\bibitem{vogelsang}
  L.E. Gordon and W. Vogelsang, \PRD {\bf 50}, 1901 (1994).
\bibitem{phenix_directgammapp}
  S.S. Adler \etal, \PRD {\bf 71}, 071102(R) (2005).
\bibitem{pythia_tuned}
  We used PYTHIA 6.205 with a modified set of parameters~\cite{phenix_auau130} 
  and CTEQ5L parton distribution functions~\cite{cteq5l}.
\bibitem{cteq5l}
  H.H. Lai \etal, Eur. Phys. J. {\bf C 12}, 375 (2000).
\bibitem{cacciari05}
  M. Cacciari, P. Nason, and R. Vogt, hep-ph/0502203.
\bibitem{pdf}
  M. Gl\"uck, E. Reya, and A. Vogt, Eur. Phys. J. {\bf C 5}, 461 (1998);
  A.D. Martin, R.G Roberts, W.J. Stirling, and R.S. Thorne, 
  Eur. Phys. J. {\bf C 4}, 463 (1998).
\end{references}

\end{document}

%
% ****** End of file template4.tex ******
